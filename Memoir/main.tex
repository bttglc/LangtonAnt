\documentclass{article}
\usepackage[utf8]{inputenc}

\title{Langton's ant \\  \LARGE Université Sorbonne Paris Nord \\ \large Initiation à la recherche en informatique}
\author{BENHADID Ayman, BOTTIGLIERI Gianluca, \\ BOULFRAD Imam, BOUZERAR Naila,  DJEROUA Mustapha}
\date{\today}

\begin{document}

\maketitle

\newpage

\tableofcontents

\newpage

\tableofcontents

\section{Introduction}
\subsection{Overview}
Langton's ant is a two-dimensional universal Turing machine with a very simple set of rules but complex emergent behavior. 
An "ant" moves on a grid of cells that can be either white or black: depending on the color of the cell it stands on, the ant 
inverts the color of the cell it stands on at the start and then moves in a certain direction or in the opposite one depending
on the original color of the starting cell. 
The first few thousands moves produce simple patterns that are often symmetric and repetitive, but after around 11000 moves 
the ant starts moving in a repetitive manner, creating a "highway" pattern that keeps repeating itself indefinitely.

\subsection{Historical background}
Langton's ant was invented by Chris Langton in 1986 while he was a graduate student at the University of Michigan and it is one of the first 
and most famous examples of artificial life: in the 40 years that followed, it gave birth to multiple  variations, ranging from version with 
multiple colors, to multiple ants, to different shapes of grids and cells, and it's still object of research and debate among computer scientist.
The field of artificial life, of which Langton himself is considered one of the founders, brought to the light concepts like 
the one of cellular automata, mathematical models that nowadays are used to model complicated, non-linear, systems, in fields including but not limited to physics,
chemistry, biology, meteorology, cosmology, computational science and engineering. By asking questions about the nature of life and the possible
behaviours life forms could have, simulating them, observing patterns and formulating hypothesis based on them, scientists have managed to 
advance understanding of actual, real life forms dramatically. For example, we now know that plants regulate their intake and loss of gases 
via a cellular automaton mechanism where each stoma on the leaf acts as a cell.

\subsection{Contents of this report}
The goal of this paper is to give the reader the tools to summarily understand the problem to then introduce an overview of the state of the art 
of Langton's ant and to finally present the results of an experiment we conducted on a finite, all-white starting grid with wrap-around properties.
The main goal of our explorative work was to observe the properties of the problem in a more limited and less studied scenario to try and derive some interesting
conclusions, familiarising with rigorous formulation of properties, simulations, data collection, data analysis and hypothesis formulation.

Section 2 consists of a series of simple explaination of fundamental concepts behind the problem.

Section 3 is a brief overview of the state of the art of Langton's ant, with a final part focused on the most recent findings and open questions.

Section 4 is an explanation of the proof of the universality of Langton's ant.

Section 5 is the core of the paper, where we present our experiment, the results and the analysis of the results.

Section 6 is a brief overview of the code we used to conduct the experiment.

Section 7 is a brief overview of related topics.

Section 8 constitutes the conclusion of the paper and includes our hypothesis and further questions.

\section{Technical prerequisite topics}
\subsection{Turing machine}
\subsubsection{Universality}
\subsection{Artificial life}
\subsubsection{Cellular automaton}
\subsection{Entropy}
\subsection{Attractor}

\section{State of the art}
\subsection{Analysis of behaviour}
\subsection{Algorithmic developments}
\subsection{Recent findings}
\subsection{Open questions}

\section{Proof of universality}

\section{Explorative work}
\subsection{Problem on finite grids with wrap-around}
\subsection{Reason for choice}
\subsection{Related concepts}
\subsubsection{Torus}
\subsubsection{Coupon collector}
\subsubsection{Markov chain}
\subsection{Experiment}
\subsection{Results}
\subsection{Analysis of results and discussion}

\section{Code of experiment}
\subsection{GUI code}
\subsection{Simulation code}
\subsection{Plotting code}

\section{Related topics}
\subsection{Other cellular automatons}
\subsection{Variations}
\subsubsection{Multiple ants}
\subsubsection{Multiple colors}
\subsubsection{Different grids}


\section{Conclusion}


\section{Bibliography}

\end{document}

\documentclass{article}
\usepackage[utf8]{inputenc}

\title{Langton's ant \\ \large Initiation à la recherche en informatique}
\author{Gianluca BOTTIGLIERI, Naila BOUZERAR, \\ Mustapha DJEROUA, Name SURNAME, Name SURNAME}
\date{\today}

\begin{document}

\maketitle

\newpage

\tableofcontents

\newpage

\tableofcontents

\section{Introduction}
\subsection{Overview}
\subsection{Contents of the report}

\section{Technical prerequisite topics}
\subsection{Turing machine}
\subsubsection{Universality}
\subsubsection{Cellular automaton}
\subsection{Entropy}
\subsection{Attractor}

\section{State of the art}
\subsection{Historical background}
\subsection{Analysis of behaviour}
\subsection{Algorithmic developments}
\subsection{Recent findings}
\subsection{Open questions}

\section{Related topics}
\subsection{Other cellular automatons}
\subsection{Variations}
\subsubsection{Multiple ants}
\subsubsection{Multiple colors}
\subsubsection{Different grids}
\subsection{Busy beaver problem}

\newpage

\section{Explorative work}
\subsection{Problem on finite grids with wrap-around}
\subsection{Reason for choice}
\subsection{Related concepts}
\subsection{Torus}
\subsubsection{Hamiltonian path}
\subsubsection{Coupon collector}
\subsubsection{Markov chain}
\subsection{Experiment}
\subsection{Results}
\subsection{Analysis of results and discussion}

\section{Conclusion}


\section{Annexes}
\subsection{Graphs from case study}

\end{document}
